\documentclass[12pt]{article}
\title{Tudnivalok}
\author{Visontai Dávid}
\usepackage[utf8]{inputenc}
\usepackage[T1]{fontenc}
\usepackage[magyar]{babel}
\begin{document}

\subsection*{Tudnivalók a Kondenzált anyagok fizikája gyakorlathoz }
\indent   
A kurzus a \emph{Kondenzált anyagok fizikája}  című, a fizika BSc 3. évfolyamos hallgatóinak szóló tárgyhoz tartozó gyakorlat. A heti 1 órában tartott gyakorlati órák feladata az előadáson ismertetett anyag elmélyítése, az ahhoz kapcsolódó feladatok megoldásával, a nehezebb elméleti részek átismétlésével és néhány, az előadás időkeretébe belenem férő szemelvény áttekintésével (pl. a szoros kötésű közelítés). A fentieken kívül a gyakorlat célja, hogy a korábban tanult, itt alkalmazásra kerülő fizikai és matematikai ismereteket (pl. feszültségtenzor, diffrakció, rezgések sajátérték-problémája, Fourier-transzformáció) felelevenítsük.
A gyakorlati órákon részt venni nem kötelező (névsor, katalógus nincs), de ajánlott, mert az anyag otthoni feldolgozásához és a számonkérésre való felkészüléshez szükség lesz a gyakorlati jegyzetre. 
\subsubsection*{Minden információ a \textbf{benedek.web.elte.hu} honlapon lesz fenn.}

\subsection*{Zárthelyi dolgozatok (ZH-k)}
A számonkérésre a félév során két zárthelyi dolgozat formájában kerül sor, amelyek elsősorban az anyag megértését ellenőrzik. A dolgozatok időpontja a gyakorlaton kerül egyeztetésre, de várhatóan a 10-ik és a 15-ik héten lesznek. Mindkét ZH-n külön érdemjegyet lehet szerezni (1-től 5-ig) az elért pontszám százalékos aránya szerint:
\begin{itemize}
\item 100\%-80\%: jeles (5)
\item 79\%-67\%: jó (4)
\item 66\%-50\%: közepes (3)
\item 49\%-35\%: elégséges (2)
\item 34\%-0\%: elégtelen (1)
\end{itemize}

\paragraph*{Fontos! A gyakorlat sikeres teljesítéséhez mindkét ZH-ból legalább 2-es (elégséges) érdemjegy megszerzése szükséges!}
A gyakorlatot sikeresen teljesítők félév végi érdemjegyét a két ZH-n elért átlagos százalékos teljesítmény határozza meg.
A ZH-kon író- és rajzeszközökön, illetve számológépen kívül semmilyen segédeszköz (könyv, füzet, jegyzet, mobil-eszköz, szomszéd, másodszomszéd stb.) nem használható. 

\paragraph{A ZH-k vázlatos tematikája a következő:}
\begin{enumerate}
\item  ZH, A kristályok szerkezete: hosszú távú rend, röntgendiffrakció, diszlokációk;
\item  ZH, Rácsrezgések és elektronok: rácsrezgések 1, 2 és 3 dimenzióban, fononok, közel szabad elektronok modellje, szoros kötésű közelítés;
\end{enumerate}
Ha valaki egyik vagy mindkét zárthelyi dolgozatot elmulasztotta vagy elégtelenre írta meg, esetleg javítani szeretne, annak lehetősége van megírni a pótzárthelyiket (mindkét ZH-ból külön). Ezek eredménye felülírja az eredeti eredményt.

\paragraph{Tudnivalók az utóvizsgáról (UV) }
Akinek a zárthelyi dolgozatok és a pótzárthelyik során nem sikerül a gyakorlati jegy megszerzése, lehetősége van utóvizsgán a javításra. (Ezt az opciót csak végső esetben ajánlom, mert a felkészülés nehezebb.) A vizsgaalkalom a Neptunban lesz meghirdetve, ennek felvétele az UV előfeltétele. Az UV-n elégtelen és elégséges érdemjegy szerezhető a pótlandó zárthelyikre.

\paragraph{Kapcsolat:}
Bármilyen észrevétellel, kérdéssel nyugodtan lehet engem keresni
e-mailen a \textbf{david.visontai@complex.elte.hu} címen, vagy személyesen az \textbf{ÉT 5.134}-ben.
\\
 
\vfill
\begin{tabular}{@{}p{2.5in}p{2in}@{}}
& Visontai Dávid \\
& Gyakorlatvezető\\
& Anyagfizika Tanszék\\
& \\
& \\
\end{tabular}



\end{document}
