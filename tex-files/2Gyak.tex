\documentclass[12pt]{article}

\usepackage{amsmath}
\usepackage[utf8]{inputenc}
\usepackage[T1]{fontenc}
\usepackage[magyar]{babel}
\usepackage{graphicx}
\usepackage{float}
\usepackage[paper=a4paper,margin=1in]{geometry}

\usepackage{siunitx}
%%\title{%
%%  Kondenzált anyagok fizikája \\
%%  \large 1. Gyakorlat}
%%\author{}  
%%\maketitle


\begin{document}


\centerline{
\textsc{\Large{ Kondenzált anyagok fizikája}}
}
\centerline{ 
\textsc{\large{1. Gyakorlat}}
}
\vspace{10mm}
\textbf{Gy1.} Egy korábbi Kömal feladat: Andi és Bandi azon vitatkoznak, hogy ha egy nemesgázt lehűtünk olyan alacsony hőmérsékletre, hogy megszilárdul, milyen típusú kristályrács alakul ki benne. Abban egyetértenek, hogy a lehető legsűrűbb elrendeződés fog kialakulni.
Andi azt mondja, hogy az egymással párhuzamos síkokban négyzetrácsban helyezkednek el az atomok, és bármelyik rácssík atomjai az alattuk lévő rácssík atomjai közötti „hézagokban ülnek”.
Bandi azt mondja, hogy ő ezt a „hézagokban ülő”  elvet már a sík
on belül is alkalmazná, s így nem négyzetrácsba, hanem szabályos háromszögrácsba rendezné el az atomokat.
Kinek van igaza? Melyik a sűrűbb elrendeződés? Mekkora a szabályos gömböknek tekinthető atomok „térkitöltése” az $A$, illetve a $B$ esetben?
\\
\\

\textbf{Gy2.} Határozzuk meg a gyémánt kitöltési tényezőjét! Milyen a gyémánt reciprokrácsa?
\\

\textbf{Gy3.} A grafén a grafit egyetlen kristálysíkja, melyben a szénatomok szabályos kétdimenziós hatszögrácsot alkotva helyezkednek el. Adjuk meg a grafén reciprokrácsát, elemi celláját és az elemi cellában lévő atomok számát, ha a szomszédos szénatomok távolsága $a$.
\\
\\\\
\textbf{Nehéz feladat:}
\\

\textbf{Gy4.} A berillium szoros pakolású hatszöges kristályráccsal ren
delkezik. Határozzuk meg a követ-kezőket:
a) Wigner–Seitz-cella térfogatát;
b) reciprokrácsot;
c) Brillouin-zóna térfogatát.



\end{document}