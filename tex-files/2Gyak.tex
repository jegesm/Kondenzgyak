\documentclass[12pt]{article}

\usepackage{amsmath}
\usepackage[utf8]{inputenc}
\usepackage[T1]{fontenc}
\usepackage[magyar]{babel}
\usepackage{graphicx}
\usepackage{float}
\usepackage[paper=a4paper,margin=1in]{geometry}

\usepackage{siunitx}
%%\title{%
%%  Kondenzált anyagok fizikája \\
%%  \large 1. Gyakorlat}
%%\author{}  
%%\maketitle


\begin{document}


\centerline{
\textsc{\Large{ Kondenzált anyagok fizikája}}
}
\centerline{ 
\textsc{\large{2. Gyakorlat}}
}
\vspace{10mm}

\textbf{Gy1.} Határozzuk meg a gyémánt kitöltési tényezőjét! Milyen a gyémánt reciprokrácsa?
\\

\textbf{Gy2.} A grafén a grafit egyetlen kristálysíkja, melyben a szénatomok szabályos kétdimenziós hatszögrácsot alkotva helyezkednek el. Adjuk meg a grafén reciprokrácsát, elemi celláját és az elemi cellában lévő atomok számát, ha a szomszédos szénatomok távolsága $a$.
\\

\textbf{Gy3. (nehéz)} A berillium szoros pakolású hatszöges kristályráccsal rendelkezik. Határozzuk meg a követ-kezőket:
\\
a) Wigner–Seitz-cella térfogatát;
\\
b) reciprokrácsot;
\\
c) Brillouin-zóna térfogatát.
\\

\textbf{Gy4. (nehéz)} (Ashcroft-Mermin Problems 5.2) Mutassuk meg, hogy a hexagonális Bravais-rács reciprok rácsa is hexagonális


\end{document}